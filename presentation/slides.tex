\documentclass[xcolor={dvipsnames}]{beamer}

\usepackage[french]{babel}
\usepackage{datetime}
\usepackage{graphicx}

\usetheme{CambridgeUS}

\setbeamercolor*{purple}{bg=Fuchsia,fg=Goldenrod}
\setbeamercolor*{yellow}{bg=Goldenrod,fg=Fuchsia}
\setbeamercolor*{palette primary}{use=yellow,fg=yellow.fg,bg=yellow.bg}
\setbeamercolor*{palette secondary}{use=yellow,fg=yellow.fg!75,bg=yellow.bg!75}
\setbeamercolor*{palette tertiary}{use=purple,fg=purple.fg,bg=purple.bg}
\setbeamercolor*{block title alerted}{use=purple,fg=purple.fg,bg=purple.bg}
\setbeamercolor*{block body alerted}{use=yellow, fg=yellow.fg,bg=yellow.bg!80}
\setbeamercolor*{titlelike}{use=yellow,fg=yellow.fg, bg=yellow.bg}
\setbeamercolor*{frametitle}{use=yellow, bg=yellow.bg!20,fg=yellow.fg}
\setbeamercolor*{structure}{use=purple, fg=purple.fg, bg=purple.bg}
\setbeamercolor*{section number projected}{use=yellow, fg=yellow.fg, bg=yellow.bg}
\setbeamercolor*{section in toc}{use= yellow, bg=yellow.bg, fg=yellow.fg}
\setbeamercolor*{normal text}{use=yellow, bg=yellow.bg!10, fg=yellow.fg!50!black}



\title{Processeur RISC-V}
\author[Farsi, Bennaoum, Landreau, Dischert]{Assim Farsi \and Séphora Bennaoum \and Félix Landreau \and Hugo Dischert}
\newdate{date}{27}{01}{2026}
\date{\displaydate{date}}
\institute[]{École normale supérieure}

\begin{document}

\frame{\titlepage}

\section{Introduction}
\begin{frame}{Plan}
\tableofcontents
\end{frame}

\section{Architecture du processeur}
\begin{frame}{Composants du processeur} 
Le processeur manipule des mots de 32 bits et est petit-boutiste.
\begin{enumerate}
  \item ALU
  \item Décodeur d'instructions
  \item 32 registres classiques, plus un registre pour le program counter
  \item RAM avec granularité de 4 octets
  \item Multiplieur et diviseur
\end{enumerate}
\end{frame}

\begin{frame}{ALU}

\begin{alertblock}{Flag inférieur}
  \(NF = (SF \lor (a[n-1] \land \lnot b[n-1]) ) \land (a[n-1] \lor \lnot b[n-1])\)
\end{alertblock}
\end{frame}

\begin{frame}{RAM et instructions}

\end{frame}

\begin{frame}{Multplieur et diviseur}
\begin{alertblock}{Equations du multiplieur}
  Non-signé : \( \text{MUL(a,b)} = a_0\times b + 2\times a_{[1:]}\times b \)\\
  Signé :
  \begin{align*}
    \text{MUL(a,b)} = (-2^{n-1}a_{n-1} + a_{[0:n-2]})\times (-2^{n-1}b_{n-1} + b_{[0:n-2]})\\  
                    = 2^{2n-2}a_{n-1}b_{n-1} + a_{[0:n-2]}\times b_{[0:n-2]} - 2^{n-1}(a_{n-1}b_{[0:n-2]} + b_{n-1}a_{[0:n-2]})
  \end{align*}
\end{alertblock}
\end{frame}

\section{Compilation et décodage d'instructions}

\begin{frame}{Compilateur}
Le compilateur est divisé en deux parties : un lexer/parser qui lit les fichiers assembleurs, et qui apppelle la librairie qui donne le code binaire des instructions.

Lexer et parser codés à la main en python.
Choix techniques : 
hyppothèse de gestion des labels : une instruction = une ligne de code produit

Gestion des labels '1'-'9'


\end{frame}

\begin{frame}{Décodeur d'instructions}
Les instructions sont représentées selon les conventions RISC-V :
\includegraphics[width = 0.7\textwidth]{instruction_formats.png}
\end{frame}

\section{Choix techniques faits}
\begin{frame}{Pipeline} %Peut être à mettre dans la section au dessus
\end{frame}

\begin{frame}{Simulation de périphériques}
Pour pouvoir afficher des nombres et avoir accès au temps, le processeur peut accéder à travers la RAM à des périphériques, simulés par le simulateur de netlist. Ces périphériques sont:
\begin{itemize}
\item Afficheur de date et heure
\item Horloge donnant le temps Unix
\end{itemize}
\end{frame}

\section{Horloge}
\begin{frame}{Fonctionnement de l'horloge}
L'horloge est écrit en assembleur et compilée par notre compilateur. Elle fait usage de l'ensemble des fonctionnalités de notre processeur: ALU, branchements conditionnels et inconditionnels, utilisation de nombreux registres, du multplieur, du diviseur, des labels du compilateur, et simulation de périphériques.

\end{frame}

\begin{frame}{Démonstration de l'horloge}

\end{frame}

\section{Conclusion}
\begin{frame}{Conclusion}
\end{frame}
\end{document}
